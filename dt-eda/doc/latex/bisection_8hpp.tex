\hypertarget{bisection_8hpp}{
\section{bisection.hpp File Reference}
\label{bisection_8hpp}\index{bisection.hpp@{bisection.hpp}}
}
Header file for \hyperlink{bisection_8cpp}{bisection.cpp}. 

{\tt \#include \char`\"{}eda.hpp\char`\"{}}\par
\subsection*{Functions}
\begin{CompactItemize}
\item 
void \hyperlink{bisection_8hpp_d0ddd2396f584f233c8946e2fd5a6e79}{bisection} (\hyperlink{struct_parameters}{Parameters} $\ast$params, \hyperlink{struct_average_population_statistics}{Average\-Population\-Statistics} $\ast$avg\_\-stats)
\begin{CompactList}\small\item\em Bisection method for determining optimal population size. \item\end{CompactList}\item 
int \hyperlink{bisection_8hpp_495b426d03f99a23286cb2cdfce56fd2}{do\_\-runs} (\hyperlink{struct_parameters}{Parameters} $\ast$params, \hyperlink{struct_average_population_statistics}{Average\-Population\-Statistics} $\ast$avg\_\-stats, int quiet=1)
\begin{CompactList}\small\item\em Run a number of successful runs, terminate on a failure. \item\end{CompactList}\item 
void \hyperlink{bisection_8hpp_66b0ce2c5fe3b131a38a1459e75d41d6}{print\_\-bisection\_\-summary} (FILE $\ast$f, \hyperlink{struct_average_population_statistics}{Average\-Population\-Statistics} $\ast$stats)
\begin{CompactList}\small\item\em Print summary of a set of runs (represented by average statistics). \item\end{CompactList}\end{CompactItemize}


\subsection{Detailed Description}
Header file for \hyperlink{bisection_8cpp}{bisection.cpp}. 



Definition in file \hyperlink{bisection_8hpp-source}{bisection.hpp}.

\subsection{Function Documentation}
\hypertarget{bisection_8hpp_d0ddd2396f584f233c8946e2fd5a6e79}{
\index{bisection.hpp@{bisection.hpp}!bisection@{bisection}}
\index{bisection@{bisection}!bisection.hpp@{bisection.hpp}}
\subsubsection[bisection]{\setlength{\rightskip}{0pt plus 5cm}void bisection (\hyperlink{struct_parameters}{Parameters} $\ast$ {\em params}, \hyperlink{struct_average_population_statistics}{Average\-Population\-Statistics} $\ast$ {\em avg\_\-stats})}}
\label{bisection_8hpp_d0ddd2396f584f233c8946e2fd5a6e79}


Bisection method for determining optimal population size. 



Definition at line 15 of file bisection.cpp.

References do\_\-runs(), Average\-Population\-Statistics::population\_\-size, Parameters::population\_\-size, Parameters::quiet\_\-mode, separator(), and Parameters::verbose\_\-mode.

Referenced by main().

\begin{Code}\begin{verbatim}16 {
17   AveragePopulationStatistics current_avg_stats;
18   Parameters *current_params = new Parameters;
19   int result;
20   int quiet=params->quiet_mode;
21 
22   params->quiet_mode=1;
23   params->verbose_mode=0;
24 
25   // make the first set of runs
26 
27   *current_params = *params;
28   result=do_runs(current_params,&current_avg_stats,quiet);
29   
30   int minN=0;
31   int maxN=0;
32 
33   if (result)
34     {
35       // if successful, try to halve population size until failure
36 
37       *avg_stats=current_avg_stats;
38 
39       while (result)
40         {
41           current_params->population_size/=2;
42           result=do_runs(current_params,&current_avg_stats,quiet);
43 
44           if (result)
45             *avg_stats=current_avg_stats;
46         };
47       
48       // set initial population size bounds
49 
50       minN=current_params->population_size;
51       maxN=minN*2;
52     }
53   else
54     {
55       // if failure, double population size until success
56 
57       while (!result)
58         {
59           current_params->population_size*=2;
60           result=do_runs(current_params,&current_avg_stats,quiet);
61 
62           if (result)
63             *avg_stats=current_avg_stats;
64         };
65       
66       // set initial population size bounds
67 
68       maxN=current_params->population_size;
69       minN=maxN/2;
70     }
71 
72   // perform bisection until interval width is at most 5% of the
73   // lower bound
74 
75   while ((((double)maxN-minN)/minN)>0.05)
76     {
77       // try to make a set of runs in the middle of the interval
78 
79       current_params->population_size=(minN+maxN)/2;
80       result=do_runs(current_params,&current_avg_stats,quiet);
81 
82       if (result)
83         *avg_stats=current_avg_stats;
84 
85       // success goes halves the interval downwards, failure upwards
86 
87       if (result)
88         maxN=current_params->population_size;
89       else
90         minN=current_params->population_size;
91     }
92 
93   if (quiet==0)
94     {
95       printf("Finished bisection with N=%u\n",avg_stats->population_size);
96       separator(stdout);
97     };
98 
99   delete current_params;
100 }
\end{verbatim}\end{Code}


\hypertarget{bisection_8hpp_495b426d03f99a23286cb2cdfce56fd2}{
\index{bisection.hpp@{bisection.hpp}!do_runs@{do\_\-runs}}
\index{do_runs@{do\_\-runs}!bisection.hpp@{bisection.hpp}}
\subsubsection[do\_\-runs]{\setlength{\rightskip}{0pt plus 5cm}int do\_\-runs (\hyperlink{struct_parameters}{Parameters} $\ast$ {\em params}, \hyperlink{struct_average_population_statistics}{Average\-Population\-Statistics} $\ast$ {\em avg\_\-stats}, int {\em quiet} = {\tt 1})}}
\label{bisection_8hpp_495b426d03f99a23286cb2cdfce56fd2}


Run a number of successful runs, terminate on a failure. 



Definition at line 107 of file bisection.cpp.

References average\_\-population\_\-statistics(), Parameters::num\_\-bisection\_\-runs, one\_\-run(), and Parameters::population\_\-size.

Referenced by bisection().

\begin{Code}\begin{verbatim}108 {
109   PopulationStatistics *stats = new PopulationStatistics[params->num_bisection_runs];
110   int run;
111   int failed=0;
112   
113   for (run=0; (run<params->num_bisection_runs)&&(!failed); run++)
114     {
115       int result=one_run(params,&stats[run]);
116       failed=!result;
117     };
118   
119   if (quiet==0)
120     printf("bisection -> %2u / %2u successes with N=%u\n",
121            run,params->num_bisection_runs,params->population_size);
122   
123   average_population_statistics(avg_stats,stats,params->num_bisection_runs);
124 
125   delete[] stats;
126 
127   if (failed)
128     return 0;
129   else
130     return 1;
131 }
\end{verbatim}\end{Code}


\hypertarget{bisection_8hpp_66b0ce2c5fe3b131a38a1459e75d41d6}{
\index{bisection.hpp@{bisection.hpp}!print_bisection_summary@{print\_\-bisection\_\-summary}}
\index{print_bisection_summary@{print\_\-bisection\_\-summary}!bisection.hpp@{bisection.hpp}}
\subsubsection[print\_\-bisection\_\-summary]{\setlength{\rightskip}{0pt plus 5cm}void print\_\-bisection\_\-summary (FILE $\ast$ {\em f}, \hyperlink{struct_average_population_statistics}{Average\-Population\-Statistics} $\ast$ {\em stats})}}
\label{bisection_8hpp_66b0ce2c5fe3b131a38a1459e75d41d6}


Print summary of a set of runs (represented by average statistics). 



Definition at line 138 of file bisection.cpp.

References Average\-Population\-Statistics::avg\_\-max\-F, Average\-Population\-Statistics::avg\_\-num\_\-evals, Average\-Population\-Statistics::num\_\-runs, and Average\-Population\-Statistics::population\_\-size.

Referenced by main().

\begin{Code}\begin{verbatim}139 {
140   fprintf(f,"Bisection summary output:\n");
141   fprintf(f,"   num_runs        = %u\n",stats->num_runs);
142   fprintf(f,"   avg_best_found  = %f\n",stats->avg_maxF);
143   fprintf(f,"   avg_num_evals   = %f\n",stats->avg_num_evals);
144   fprintf(f,"   pop_size        = %u\n",stats->population_size);
145 }
\end{verbatim}\end{Code}


