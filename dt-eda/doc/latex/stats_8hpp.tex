\hypertarget{stats_8hpp}{
\section{stats.hpp File Reference}
\label{stats_8hpp}\index{stats.hpp@{stats.hpp}}
}
Header file for \hyperlink{stats_8cpp}{stats.cpp}. 

\subsection*{Data Structures}
\begin{CompactItemize}
\item 
struct \hyperlink{struct_population_statistics}{Population\-Statistics}
\begin{CompactList}\small\item\em Basic population statistics (single run). \item\end{CompactList}\item 
struct \hyperlink{struct_average_population_statistics}{Average\-Population\-Statistics}
\begin{CompactList}\small\item\em Population statistics for a set of runs. \item\end{CompactList}\end{CompactItemize}
\subsection*{Functions}
\begin{CompactItemize}
\item 
void \hyperlink{stats_8hpp_a7fd66cda3e240f089d86932c946b7c2}{init\_\-population\_\-statistics} (\hyperlink{struct_population_statistics}{Population\-Statistics} $\ast$stats)
\begin{CompactList}\small\item\em Initialize the population statistics structure. \item\end{CompactList}\item 
int \hyperlink{stats_8hpp_3ddb72b17ff7c8324994fa729741596d}{compute\_\-population\_\-statistics} (int $\ast$$\ast$x, double $\ast$f, int n, int N, int $\ast$num\_\-vals, \hyperlink{struct_population_statistics}{Population\-Statistics} $\ast$stats)
\begin{CompactList}\small\item\em Compute basic population statistics like the maximum, mean, etc. \item\end{CompactList}\item 
int \hyperlink{stats_8hpp_d546d85226735a391bd6ee38b9fa9c8c}{average\_\-population\_\-statistics} (\hyperlink{struct_average_population_statistics}{Average\-Population\-Statistics} $\ast$avg, \hyperlink{struct_population_statistics}{Population\-Statistics} $\ast$stats, int num\_\-runs)
\begin{CompactList}\small\item\em Compute averages of population statistics over multiple runs. \item\end{CompactList}\end{CompactItemize}


\subsection{Detailed Description}
Header file for \hyperlink{stats_8cpp}{stats.cpp}. 



Definition in file \hyperlink{stats_8hpp-source}{stats.hpp}.

\subsection{Function Documentation}
\hypertarget{stats_8hpp_d546d85226735a391bd6ee38b9fa9c8c}{
\index{stats.hpp@{stats.hpp}!average_population_statistics@{average\_\-population\_\-statistics}}
\index{average_population_statistics@{average\_\-population\_\-statistics}!stats.hpp@{stats.hpp}}
\subsubsection[average\_\-population\_\-statistics]{\setlength{\rightskip}{0pt plus 5cm}int average\_\-population\_\-statistics (\hyperlink{struct_average_population_statistics}{Average\-Population\-Statistics} $\ast$ {\em avg}, \hyperlink{struct_population_statistics}{Population\-Statistics} $\ast$ {\em stats}, int {\em num\_\-runs})}}
\label{stats_8hpp_d546d85226735a391bd6ee38b9fa9c8c}


Compute averages of population statistics over multiple runs. 



Definition at line 74 of file stats.cpp.

References Average\-Population\-Statistics::avg\_\-avg\-F, Average\-Population\-Statistics::avg\_\-max\-F, Average\-Population\-Statistics::avg\_\-min\-F, Average\-Population\-Statistics::avg\_\-num\_\-evals, Population\-Statistics::avg\-F, Average\-Population\-Statistics::max\_\-max\-F, Population\-Statistics::max\-F, Average\-Population\-Statistics::min\_\-min\-F, Population\-Statistics::min\-F, Population\-Statistics::num\_\-evals, Average\-Population\-Statistics::num\_\-runs, Population\-Statistics::population\_\-size, Average\-Population\-Statistics::population\_\-size, Population\-Statistics::problem\_\-size, and Average\-Population\-Statistics::problem\_\-size.

Referenced by do\_\-runs().

\begin{Code}\begin{verbatim}75 {
76   // initialize the average stats
77 
78   avg->avg_minF=stats[0].minF;
79   avg->avg_maxF=stats[0].maxF;
80   avg->avg_avgF=stats[0].avgF;
81   avg->avg_num_evals=stats[0].num_evals;
82 
83   avg->min_minF=stats[0].minF;
84   avg->max_maxF=stats[0].maxF;
85 
86   avg->population_size=stats[0].population_size;
87   avg->problem_size=stats[0].problem_size;
88   avg->num_runs=num_runs;
89 
90   // do the homework
91 
92   for (int run=1; run<num_runs; run++)
93     {
94       avg->avg_minF+=stats[run].minF;
95       avg->avg_maxF+=stats[run].maxF;
96       avg->avg_avgF+=stats[run].avgF;
97       avg->avg_num_evals+=stats[run].num_evals;
98 
99       if (stats[run].minF<avg->min_minF)
100         avg->min_minF=stats[run].minF;
101 
102       if (stats[run].maxF>avg->max_maxF)
103         avg->max_maxF=stats[run].maxF;
104     }
105 
106   avg->avg_minF/=num_runs;
107   avg->avg_maxF/=num_runs;
108   avg->avg_avgF/=num_runs; 
109   avg->avg_num_evals/=num_runs;
110 
111   // return the number of processed runs
112 
113   return num_runs;
114 }
\end{verbatim}\end{Code}


\hypertarget{stats_8hpp_3ddb72b17ff7c8324994fa729741596d}{
\index{stats.hpp@{stats.hpp}!compute_population_statistics@{compute\_\-population\_\-statistics}}
\index{compute_population_statistics@{compute\_\-population\_\-statistics}!stats.hpp@{stats.hpp}}
\subsubsection[compute\_\-population\_\-statistics]{\setlength{\rightskip}{0pt plus 5cm}int compute\_\-population\_\-statistics (int $\ast$$\ast$ {\em x}, double $\ast$ {\em f}, int {\em n}, int {\em N}, int $\ast$ {\em num\_\-vals}, \hyperlink{struct_population_statistics}{Population\-Statistics} $\ast$ {\em stats})}}
\label{stats_8hpp_3ddb72b17ff7c8324994fa729741596d}


Compute basic population statistics like the maximum, mean, etc. 



Definition at line 15 of file stats.cpp.

References Population\-Statistics::avg\-F, Population\-Statistics::idx\_\-max\-F, Population\-Statistics::idx\_\-min\-F, Population\-Statistics::max\-F, Population\-Statistics::min\-F, Population\-Statistics::population\_\-size, and Population\-Statistics::problem\_\-size.

Referenced by one\_\-run().

\begin{Code}\begin{verbatim}16 {
17   // store population size and problem size
18 
19   stats->population_size=N;
20   stats->problem_size=n;
21 
22   // compute basic fitness statistics
23 
24   stats->idx_minF = stats->idx_maxF = 0;
25   stats->minF = stats->maxF = stats->avgF = f[0];
26 
27   for (int i=1; i<N; i++)
28     {
29       stats->avgF+=f[i];
30 
31       if (f[i]>stats->maxF)
32         {
33           stats->maxF=f[i];
34           stats->idx_maxF=i;
35         }
36       else
37         if (f[i]<stats->minF)
38           {
39             stats->minF=f[i];
40             stats->idx_minF=i;
41           }
42     }
43 
44   stats->avgF/=N;
45 
46   // check whether we found the optimum
47 
48   if (is_optimal(x[stats->idx_maxF],n,num_vals))
49     stats->success=1;
50 
51   // get back
52 
53   return 0;
54 }
\end{verbatim}\end{Code}


\hypertarget{stats_8hpp_a7fd66cda3e240f089d86932c946b7c2}{
\index{stats.hpp@{stats.hpp}!init_population_statistics@{init\_\-population\_\-statistics}}
\index{init_population_statistics@{init\_\-population\_\-statistics}!stats.hpp@{stats.hpp}}
\subsubsection[init\_\-population\_\-statistics]{\setlength{\rightskip}{0pt plus 5cm}void init\_\-population\_\-statistics (\hyperlink{struct_population_statistics}{Population\-Statistics} $\ast$ {\em stats})}}
\label{stats_8hpp_a7fd66cda3e240f089d86932c946b7c2}


Initialize the population statistics structure. 



Definition at line 61 of file stats.cpp.

References Population\-Statistics::avg\-F, Population\-Statistics::idx\_\-max\-F, Population\-Statistics::idx\_\-min\-F, Population\-Statistics::max\-F, Population\-Statistics::min\-F, Population\-Statistics::num\_\-evals, and Population\-Statistics::success.

Referenced by one\_\-run().

\begin{Code}\begin{verbatim}62 {
63   stats->num_evals=0;
64   stats->minF=stats->maxF=stats->avgF=0;
65   stats->idx_minF=stats->idx_maxF=-1;
66   stats->success=0;
67 }
\end{verbatim}\end{Code}


