\hypertarget{heap_8hpp}{
\section{heap.hpp File Reference}
\label{heap_8hpp}\index{heap.hpp@{heap.hpp}}
}
Header file for \hyperlink{heap_8cpp}{heap.cpp}. 

\subsection*{Defines}
\begin{CompactItemize}
\item 
\#define \hyperlink{heap_8hpp_423b8ed0ab54ed8b378e0871c4bb54a1}{HEAP\_\-MINUS\_\-INFINITY}~-1
\end{CompactItemize}
\subsection*{Functions}
\begin{CompactItemize}
\item 
int \hyperlink{heap_8hpp_57de289144d3c948d6fcd8bb469e66f7}{max\_\-heapify} (double $\ast$x, int i, int n, int $\ast$index, int $\ast$rev\_\-index)
\begin{CompactList}\small\item\em Run max-heapify on a specified element of a heap (push down). \item\end{CompactList}\item 
int \hyperlink{heap_8hpp_9a7b5521ec847bff1af3dbe524f0b376}{build\_\-max\_\-heap} (int $\ast$a, int $\ast$b, double $\ast$x, int n, int $\ast$index, int $\ast$rev\_\-index)
\begin{CompactList}\small\item\em Build a max-heap for a given array. \item\end{CompactList}\item 
int \hyperlink{heap_8hpp_668e138412dbab3baf4e87d543f90870}{float\_\-up\_\-max\_\-heap} (double $\ast$x, int i, int n, int $\ast$index, int $\ast$rev\_\-index)
\begin{CompactList}\small\item\em Float up an element up the heap (after increasing its value). \item\end{CompactList}\item 
int \hyperlink{heap_8hpp_d33e06bba2950d85f9aca3f1a47d2a8e}{increase\_\-key\_\-max\_\-heap} (double $\ast$x, int i, double val, int n, int $\ast$index, int $\ast$rev\_\-index)
\begin{CompactList}\small\item\em Increase a given key in the heap. \item\end{CompactList}\item 
int \hyperlink{heap_8hpp_9d9bfd1029dc45808a4eb11ef7e4d6fa}{pop\_\-max\_\-heap} (double $\ast$x, int \&n, int $\ast$index, int $\ast$rev\_\-index)
\begin{CompactList}\small\item\em Pop the maximum (includes the removal of maximum). \item\end{CompactList}\item 
void \hyperlink{heap_8hpp_4e2ff0b6f88983d968a8c17562fa7e19}{print\_\-max\_\-heap} (double $\ast$x, int n, int $\ast$index, int $\ast$rev\_\-index)
\begin{CompactList}\small\item\em Print the max heap (used mostly for debugging). \item\end{CompactList}\item 
int \hyperlink{heap_8hpp_33f6dccbcfc6a403dce7e8ab78a69af8}{check\_\-max\_\-heap} (double $\ast$x, int n, int $\ast$index, int $\ast$rev\_\-index)
\begin{CompactList}\small\item\em Check the max heap for errors (used mostly in debugging). \item\end{CompactList}\end{CompactItemize}


\subsection{Detailed Description}
Header file for \hyperlink{heap_8cpp}{heap.cpp}. 



Definition in file \hyperlink{heap_8hpp-source}{heap.hpp}.

\subsection{Define Documentation}
\hypertarget{heap_8hpp_423b8ed0ab54ed8b378e0871c4bb54a1}{
\index{heap.hpp@{heap.hpp}!HEAP_MINUS_INFINITY@{HEAP\_\-MINUS\_\-INFINITY}}
\index{HEAP_MINUS_INFINITY@{HEAP\_\-MINUS\_\-INFINITY}!heap.hpp@{heap.hpp}}
\subsubsection[HEAP\_\-MINUS\_\-INFINITY]{\setlength{\rightskip}{0pt plus 5cm}\#define HEAP\_\-MINUS\_\-INFINITY~-1}}
\label{heap_8hpp_423b8ed0ab54ed8b378e0871c4bb54a1}




Definition at line 9 of file heap.hpp.

Referenced by Tree\-Model::learn\-Structure().

\subsection{Function Documentation}
\hypertarget{heap_8hpp_9a7b5521ec847bff1af3dbe524f0b376}{
\index{heap.hpp@{heap.hpp}!build_max_heap@{build\_\-max\_\-heap}}
\index{build_max_heap@{build\_\-max\_\-heap}!heap.hpp@{heap.hpp}}
\subsubsection[build\_\-max\_\-heap]{\setlength{\rightskip}{0pt plus 5cm}int build\_\-max\_\-heap (int $\ast$ {\em a}, int $\ast$ {\em b}, double $\ast$ {\em x}, int {\em n}, int $\ast$ {\em index}, int $\ast$ {\em rev\_\-index})}}
\label{heap_8hpp_9a7b5521ec847bff1af3dbe524f0b376}


Build a max-heap for a given array. 



Definition at line 19 of file heap.cpp.

References max\_\-heapify().

Referenced by Tree\-Model::learn\-Structure().

\begin{Code}\begin{verbatim}20 {
21   // builds a max-heap by max-heapifying all nodes with some children
22 
23   for (int i=(n>>1)-1; i>=0; i--)
24     max_heapify(x,i,n,index,rev_index);
25 
26   // get back
27 
28   return n;
29 };
\end{verbatim}\end{Code}


\hypertarget{heap_8hpp_33f6dccbcfc6a403dce7e8ab78a69af8}{
\index{heap.hpp@{heap.hpp}!check_max_heap@{check\_\-max\_\-heap}}
\index{check_max_heap@{check\_\-max\_\-heap}!heap.hpp@{heap.hpp}}
\subsubsection[check\_\-max\_\-heap]{\setlength{\rightskip}{0pt plus 5cm}int check\_\-max\_\-heap (double $\ast$ {\em x}, int {\em n}, int $\ast$ {\em index}, int $\ast$ {\em rev\_\-index})}}
\label{heap_8hpp_33f6dccbcfc6a403dce7e8ab78a69af8}


Check the max heap for errors (used mostly in debugging). 



Definition at line 156 of file heap.cpp.

\begin{Code}\begin{verbatim}157 {
158   int ok=1;
159 
160   for (int i=0; (i<n)&&(ok); i++)
161     {
162       if (i*2+1<n)
163         if (x[index[i]]<x[index[2*i+1]])
164           ok=0;
165 
166       if (i*2+2<n)
167         if (x[index[i]]<x[index[2*i+2]])
168           ok=0;
169     };
170 
171   if (ok==0)
172     printf("not OK\n");
173   else
174     printf("OK\n");
175 
176   return ok;
177 }
\end{verbatim}\end{Code}


\hypertarget{heap_8hpp_668e138412dbab3baf4e87d543f90870}{
\index{heap.hpp@{heap.hpp}!float_up_max_heap@{float\_\-up\_\-max\_\-heap}}
\index{float_up_max_heap@{float\_\-up\_\-max\_\-heap}!heap.hpp@{heap.hpp}}
\subsubsection[float\_\-up\_\-max\_\-heap]{\setlength{\rightskip}{0pt plus 5cm}int float\_\-up\_\-max\_\-heap (double $\ast$ {\em x}, int {\em i}, int {\em n}, int $\ast$ {\em index}, int $\ast$ {\em rev\_\-index})}}
\label{heap_8hpp_668e138412dbab3baf4e87d543f90870}


Float up an element up the heap (after increasing its value). 



Definition at line 94 of file heap.cpp.

References parent\_\-index.

Referenced by increase\_\-key\_\-max\_\-heap().

\begin{Code}\begin{verbatim}95 {
96   double this_x=x[index[i]];
97   int old_ii=index[i];
98 
99   int parent_idx=parent_index(i);
100 
101   while ((i>0)&&(this_x>x[index[parent_idx]]))
102     {
103       index[i]=index[parent_idx];
104       rev_index[index[i]]=i;
105       i=parent_idx;
106       if (i>0)
107         parent_idx=parent_index(i);
108     };
109 
110   index[i]=old_ii;
111   rev_index[index[i]]=i;
112 
113   return i;
114 };
\end{verbatim}\end{Code}


\hypertarget{heap_8hpp_d33e06bba2950d85f9aca3f1a47d2a8e}{
\index{heap.hpp@{heap.hpp}!increase_key_max_heap@{increase\_\-key\_\-max\_\-heap}}
\index{increase_key_max_heap@{increase\_\-key\_\-max\_\-heap}!heap.hpp@{heap.hpp}}
\subsubsection[increase\_\-key\_\-max\_\-heap]{\setlength{\rightskip}{0pt plus 5cm}int increase\_\-key\_\-max\_\-heap (double $\ast$ {\em x}, int {\em i}, double {\em val}, int {\em n}, int $\ast$ {\em index}, int $\ast$ {\em rev\_\-index})}}
\label{heap_8hpp_d33e06bba2950d85f9aca3f1a47d2a8e}


Increase a given key in the heap. 



Definition at line 83 of file heap.cpp.

References float\_\-up\_\-max\_\-heap().

Referenced by Tree\-Model::learn\-Structure().

\begin{Code}\begin{verbatim}84 {
85   x[i]=val;
86   return float_up_max_heap(x,rev_index[i],n,index,rev_index);
87 };
\end{verbatim}\end{Code}


\hypertarget{heap_8hpp_57de289144d3c948d6fcd8bb469e66f7}{
\index{heap.hpp@{heap.hpp}!max_heapify@{max\_\-heapify}}
\index{max_heapify@{max\_\-heapify}!heap.hpp@{heap.hpp}}
\subsubsection[max\_\-heapify]{\setlength{\rightskip}{0pt plus 5cm}int max\_\-heapify (double $\ast$ {\em x}, int {\em i}, int {\em n}, int $\ast$ {\em index}, int $\ast$ {\em rev\_\-index})}}
\label{heap_8hpp_57de289144d3c948d6fcd8bb469e66f7}


Run max-heapify on a specified element of a heap (push down). 



Definition at line 36 of file heap.cpp.

References left\_\-child\_\-index.

Referenced by build\_\-max\_\-heap(), and pop\_\-max\_\-heap().

\begin{Code}\begin{verbatim}37 {
38   double this_x=x[index[i]];
39   int old_ii=index[i];
40 
41   int done=0;
42   do {
43     int c1=left_child_index(i);
44     int c2=c1+1;
45 
46     int ic1=index[c1];
47     
48     if (c2<n)
49       {
50         int ic2=index[c2];
51         
52         if (x[ic1]<x[ic2])
53           {
54             ic1=ic2;
55             c1=c2;
56           };
57       };
58     
59     if (x[ic1]>this_x)
60       {
61         index[i]=index[c1];
62         rev_index[index[i]]=i;
63         i=c1;
64         if (left_child_index(i)>=n)
65           done=1;
66       }
67     else
68       done=1;
69     
70   } while (!done);
71   
72   index[i]=old_ii;
73   rev_index[index[i]]=i;
74 
75   return i;
76 };
\end{verbatim}\end{Code}


\hypertarget{heap_8hpp_9d9bfd1029dc45808a4eb11ef7e4d6fa}{
\index{heap.hpp@{heap.hpp}!pop_max_heap@{pop\_\-max\_\-heap}}
\index{pop_max_heap@{pop\_\-max\_\-heap}!heap.hpp@{heap.hpp}}
\subsubsection[pop\_\-max\_\-heap]{\setlength{\rightskip}{0pt plus 5cm}int pop\_\-max\_\-heap (double $\ast$ {\em x}, int \& {\em n}, int $\ast$ {\em index}, int $\ast$ {\em rev\_\-index})}}
\label{heap_8hpp_9d9bfd1029dc45808a4eb11ef7e4d6fa}


Pop the maximum (includes the removal of maximum). 



Definition at line 121 of file heap.cpp.

References max\_\-heapify().

Referenced by Tree\-Model::learn\-Structure().

\begin{Code}\begin{verbatim}122 {
123   n--;
124   int val=index[0];
125 
126   index[0]=index[n];
127   rev_index[index[n]]=0;
128   max_heapify(x,0,n,index,rev_index);
129 
130   return val;
131 }
\end{verbatim}\end{Code}


\hypertarget{heap_8hpp_4e2ff0b6f88983d968a8c17562fa7e19}{
\index{heap.hpp@{heap.hpp}!print_max_heap@{print\_\-max\_\-heap}}
\index{print_max_heap@{print\_\-max\_\-heap}!heap.hpp@{heap.hpp}}
\subsubsection[print\_\-max\_\-heap]{\setlength{\rightskip}{0pt plus 5cm}void print\_\-max\_\-heap (double $\ast$ {\em x}, int {\em n}, int $\ast$ {\em index}, int $\ast$ {\em rev\_\-index})}}
\label{heap_8hpp_4e2ff0b6f88983d968a8c17562fa7e19}


Print the max heap (used mostly for debugging). 



Definition at line 138 of file heap.cpp.

\begin{Code}\begin{verbatim}139 {
140   for (int i=0; i<n; i++)
141     {
142       printf("x[%u] = %5.3f (",i,x[index[i]]);
143       if (i*2+1<n)
144         printf("%5.3f ",x[index[2*i+1]]);
145       if (i*2+2<n)
146         printf("%5.3f ",x[index[2*i+2]]);
147       printf(")\n");
148     };
149 }
\end{verbatim}\end{Code}


