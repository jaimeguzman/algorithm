\hypertarget{main_8cpp}{
\section{main.cpp File Reference}
\label{main_8cpp}\index{main.cpp@{main.cpp}}
}
Main function. 

{\tt \#include $<$stdlib.h$>$}\par
{\tt \#include $<$stdio.h$>$}\par
{\tt \#include $<$string.h$>$}\par
{\tt \#include \char`\"{}bisection.hpp\char`\"{}}\par
{\tt \#include \char`\"{}eda.hpp\char`\"{}}\par
{\tt \#include \char`\"{}heap.hpp\char`\"{}}\par
{\tt \#include \char`\"{}parse-input.hpp\char`\"{}}\par
{\tt \#include \char`\"{}random.hpp\char`\"{}}\par
{\tt \#include \char`\"{}tree-model.hpp\char`\"{}}\par
\subsection*{Functions}
\begin{CompactItemize}
\item 
int \hyperlink{main_8cpp_3c04138a5bfe5d72780bb7e82a18e627}{main} (int argc, char $\ast$$\ast$argv)
\begin{CompactList}\small\item\em The main function. \item\end{CompactList}\end{CompactItemize}


\subsection{Detailed Description}
Main function. 



Definition in file \hyperlink{main_8cpp-source}{main.cpp}.

\subsection{Function Documentation}
\hypertarget{main_8cpp_3c04138a5bfe5d72780bb7e82a18e627}{
\index{main.cpp@{main.cpp}!main@{main}}
\index{main@{main}!main.cpp@{main.cpp}}
\subsubsection[main]{\setlength{\rightskip}{0pt plus 5cm}int main (int {\em argc}, char $\ast$$\ast$ {\em argv})}}
\label{main_8cpp_3c04138a5bfe5d72780bb7e82a18e627}


The main function. 



Definition at line 62 of file main.cpp.

References bisection(), Parameters::bisection, Parameters::max\_\-generations, Parameters::num\_\-bisection\_\-runs, one\_\-run(), param\_\-help(), parse\_\-input\_\-file(), Parameters::population\_\-size, print\_\-bisection\_\-summary(), print\_\-parameters(), print\_\-summary(), Parameters::problem\_\-size, Parameters::quiet\_\-mode, Parameters::replacement, Parameters::tournament\_\-size, and Parameters::verbose\_\-mode.

\begin{Code}\begin{verbatim}63 {
64   Parameters params;
65 
66   // set default parameter values
67 
68   params.population_size=300;
69   params.problem_size=25;
70   params.max_generations=50;
71   params.tournament_size=2;
72   params.replacement=0;
73   params.verbose_mode=0;
74   params.quiet_mode=0;
75   params.bisection=0;
76   params.num_bisection_runs=20;
77 
78   // process input arguments
79 
80   if (argc>2)
81     {
82       printf("Expected only one command-line argument.\n");
83       exit(-10);
84     }
85   else
86   if (argc==2)
87   {
88     if ((strcmp(argv[1],"--help")==0)||
89         (strcmp(argv[1],"-h")==0)||
90         (strcmp(argv[1],"/help")==0)||
91         (strcmp(argv[1],"/h")==0)||
92         (strcmp(argv[1],"-?")==0)||
93         (strcmp(argv[1],"/?")==0))
94       {
95         // help
96 
97         printf("Usage: dt-eda [parameter file name] [--help] [--version]\n\n");
98         param_help();
99         exit(0);
100       }
101     else 
102     if ((strcmp(argv[1],"--version")==0)||
103         (strcmp(argv[1],"-v")==0))
104       {
105         // version info
106 
107         printf("dt-eda-1.0\n");
108         exit(0);
109       }
110     else
111       {
112         // parameter file
113 
114         FILE *f=fopen(argv[1],"r");
115         if (f==NULL)
116           {
117             printf("ERROR: Could not open parameter file %s\n",argv[1]);
118             exit(-1);
119           }
120 
121         parse_input_file(f,&params);
122         printf("Parameter file: %s\n",argv[1]);
123       };
124   }
125   else
126     printf("No parameter file: Using default parameters\n");
127 
128   // print the parameters
129 
130   print_parameters(&params);
131 
132   // perform one run of the dependency-tree EDA with the current parameters
133   // and user defined functions (parameters from user-defined parameter files 
134   // go first)
135 
136   if (params.bisection)
137     {
138       AveragePopulationStatistics avg_stats;
139 
140       bisection(&params,&avg_stats);
141 
142       print_bisection_summary(stdout,&avg_stats);
143     }
144   else
145     {
146       PopulationStatistics stats;
147 
148       one_run(&params,&stats);
149 
150       print_summary(&stats);
151     };
152 
153   // get out
154 
155   return 0;
156 }
\end{verbatim}\end{Code}


