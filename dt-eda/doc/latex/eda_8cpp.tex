\hypertarget{eda_8cpp}{
\section{eda.cpp File Reference}
\label{eda_8cpp}\index{eda.cpp@{eda.cpp}}
}
EDA-specific functions (except for the model-related ones). 

{\tt \#include $<$stdio.h$>$}\par
{\tt \#include $<$string.h$>$}\par
{\tt \#include $<$stdlib.h$>$}\par
{\tt \#include \char`\"{}eda.hpp\char`\"{}}\par
{\tt \#include \char`\"{}random.hpp\char`\"{}}\par
{\tt \#include \char`\"{}obj-function.hpp\char`\"{}}\par
{\tt \#include \char`\"{}stats.hpp\char`\"{}}\par
{\tt \#include \char`\"{}tree-model.hpp\char`\"{}}\par
\subsection*{Typedefs}
\begin{CompactItemize}
\item 
typedef int \hyperlink{eda_8cpp_34093d951244ee1e9a4557d9135db829}{Replacement\-Method} (int $\ast$$\ast$x, int $\ast$$\ast$y, int n, int N, double $\ast$fx, double $\ast$fy)
\end{CompactItemize}
\subsection*{Functions}
\begin{CompactItemize}
\item 
int \hyperlink{eda_8cpp_8d2772174edcaff4c414433bb60c440d}{generate\_\-population} (int $\ast$$\ast$x, int n, int N, int $\ast$num\_\-vals=NULL)
\begin{CompactList}\small\item\em Generate a random population of individuals (uniform distribution). \item\end{CompactList}\item 
int \hyperlink{eda_8cpp_6691772a0d2ac34de31e1daf06e33343}{generate\_\-BB\_\-population} (int $\ast$$\ast$x, int n, int N, int k)
\begin{CompactList}\small\item\em Generate a poulation full of blocks of 0s and 1s. Only for debugging. \item\end{CompactList}\item 
int $\ast$$\ast$ \hyperlink{eda_8cpp_bad30812456a4a71d90a3612cd9793f7}{allocate\_\-population} (int n, int N)
\begin{CompactList}\small\item\em Allocate memory for a new population of specified parameters. \item\end{CompactList}\item 
void \hyperlink{eda_8cpp_a9eb525500760458e6348fa016194a73}{free\_\-population} (int $\ast$$\ast$x, int n, int N)
\begin{CompactList}\small\item\em Free memory occupied by a population of specified parameters. \item\end{CompactList}\item 
int \hyperlink{eda_8cpp_a7ca4af23fcef408ed2447f8113b3518}{evaluate\_\-population} (int $\ast$$\ast$x, double $\ast$f, int n, int N, long \&num\_\-evals)
\begin{CompactList}\small\item\em Evaluate a population using the user-specified objective function. \item\end{CompactList}\item 
int \hyperlink{eda_8cpp_b7834469dba3fdd4d1e73b368051dbae}{tournament\_\-selection} (int $\ast$$\ast$y, int $\ast$$\ast$x, double $\ast$f, int n, int N, int k)
\begin{CompactList}\small\item\em Tournament selection with replacement (k-ary tournaments). \item\end{CompactList}\item 
void \hyperlink{eda_8cpp_aea6e51089cae913316f55a552b9226c}{print\_\-population} (FILE $\ast$f, int $\ast$$\ast$x, int n, int N)
\begin{CompactList}\small\item\em Print the population to the specified file. \item\end{CompactList}\item 
int \hyperlink{eda_8cpp_f277260af3b0a29adf95a57fe3581404}{restricted\_\-tournament\_\-replacement} (int $\ast$$\ast$x, int $\ast$$\ast$y, int n, int N, double $\ast$fx, double $\ast$fy)
\begin{CompactList}\small\item\em Restricted tournament replacement (Harik, 1995) for niching. \item\end{CompactList}\item 
int \hyperlink{eda_8cpp_68956705a66180170efb2f00c4119486}{full\_\-replacement} (int $\ast$$\ast$x, int $\ast$$\ast$y, int n, int N, double $\ast$fx, double $\ast$fy)
\begin{CompactList}\small\item\em Replace the entire old population with the new population. \item\end{CompactList}\item 
int \hyperlink{eda_8cpp_1329a1829493fea528b0e690ab69f94e}{individual\_\-distance} (int $\ast$x, int $\ast$y, int n)
\begin{CompactList}\small\item\em Compute a distance between two individuals (number of non-matching chars). \item\end{CompactList}\item 
int \hyperlink{eda_8cpp_334f5d73c2118df22dd59546965be64c}{one\_\-run} (\hyperlink{struct_parameters}{Parameters} $\ast$params, \hyperlink{struct_population_statistics}{Population\-Statistics} $\ast$stats)
\begin{CompactList}\small\item\em Execute one run of a decision-tree EDA with the specified parameters. \item\end{CompactList}\item 
int \hyperlink{eda_8cpp_8eef87c9ae12bcae144b6b020865df42}{variation} (int $\ast$$\ast$sampled\_\-population, int $\ast$$\ast$selected\_\-population, \hyperlink{struct_parameters}{Parameters} $\ast$params, int $\ast$num\_\-vals)
\begin{CompactList}\small\item\em Variation operator of the decision-tree EDA (learns and samples model). \item\end{CompactList}\item 
void \hyperlink{eda_8cpp_be5265230638c7fb37140688226999e2}{print\_\-status} (int t, \hyperlink{struct_population_statistics}{Population\-Statistics} $\ast$stats)
\begin{CompactList}\small\item\em Print the status of the algorithm. \item\end{CompactList}\item 
void \hyperlink{eda_8cpp_ac7bc3c5ce16c23728f6ed1bd6534f56}{separator} (FILE $\ast$f, int type)
\begin{CompactList}\small\item\em Print a sequence of dashes to separate text output. \item\end{CompactList}\item 
void \hyperlink{eda_8cpp_83b5f930be040dac41889ae6d326a47a}{print\_\-summary} (\hyperlink{struct_population_statistics}{Population\-Statistics} $\ast$stats)
\begin{CompactList}\small\item\em Print summary of a run. \item\end{CompactList}\end{CompactItemize}


\subsection{Detailed Description}
EDA-specific functions (except for the model-related ones). 



Definition in file \hyperlink{eda_8cpp-source}{eda.cpp}.

\subsection{Typedef Documentation}
\hypertarget{eda_8cpp_34093d951244ee1e9a4557d9135db829}{
\index{eda.cpp@{eda.cpp}!ReplacementMethod@{ReplacementMethod}}
\index{ReplacementMethod@{ReplacementMethod}!eda.cpp@{eda.cpp}}
\subsubsection[ReplacementMethod]{\setlength{\rightskip}{0pt plus 5cm}typedef int \hyperlink{eda_8cpp_34093d951244ee1e9a4557d9135db829}{Replacement\-Method}(int $\ast$$\ast$x, int $\ast$$\ast$y, int n, int N, double $\ast$fx, double $\ast$fy)}}
\label{eda_8cpp_34093d951244ee1e9a4557d9135db829}




Definition at line 19 of file eda.cpp.

\subsection{Function Documentation}
\hypertarget{eda_8cpp_bad30812456a4a71d90a3612cd9793f7}{
\index{eda.cpp@{eda.cpp}!allocate_population@{allocate\_\-population}}
\index{allocate_population@{allocate\_\-population}!eda.cpp@{eda.cpp}}
\subsubsection[allocate\_\-population]{\setlength{\rightskip}{0pt plus 5cm}int$\ast$$\ast$ allocate\_\-population (int {\em n}, int {\em N})}}
\label{eda_8cpp_bad30812456a4a71d90a3612cd9793f7}


Allocate memory for a new population of specified parameters. 



Definition at line 84 of file eda.cpp.

Referenced by one\_\-run().

\begin{Code}\begin{verbatim}85 {
86   int i;
87   int **x;
88 
89   // allocate a population as a 2-dimensional int-array
90 
91   x=new int*[N];
92   for (i=0; i<N; i++)
93     x[i]=new int[n];
94 
95   // return the allocated array
96 
97   return x;
98 }
\end{verbatim}\end{Code}


\hypertarget{eda_8cpp_a7ca4af23fcef408ed2447f8113b3518}{
\index{eda.cpp@{eda.cpp}!evaluate_population@{evaluate\_\-population}}
\index{evaluate_population@{evaluate\_\-population}!eda.cpp@{eda.cpp}}
\subsubsection[evaluate\_\-population]{\setlength{\rightskip}{0pt plus 5cm}int evaluate\_\-population (int $\ast$$\ast$ {\em x}, double $\ast$ {\em f}, int {\em n}, int {\em N}, long \& {\em num\_\-evals})}}
\label{eda_8cpp_a7ca4af23fcef408ed2447f8113b3518}


Evaluate a population using the user-specified objective function. 



Definition at line 122 of file eda.cpp.

References objective\_\-function().

Referenced by one\_\-run().

\begin{Code}\begin{verbatim}123 {
124   // evaluate all individuals, one by one
125 
126   for (int i=0; i<N; i++)
127     f[i]=objective_function(x[i],n);
128 
129   // increase the number of evaluations
130 
131   num_evals+=N;
132 
133   // return the number of evaluated individuals
134 
135   return N;
136 }
\end{verbatim}\end{Code}


\hypertarget{eda_8cpp_a9eb525500760458e6348fa016194a73}{
\index{eda.cpp@{eda.cpp}!free_population@{free\_\-population}}
\index{free_population@{free\_\-population}!eda.cpp@{eda.cpp}}
\subsubsection[free\_\-population]{\setlength{\rightskip}{0pt plus 5cm}void free\_\-population (int $\ast$$\ast$ {\em x}, int {\em n}, int {\em N})}}
\label{eda_8cpp_a9eb525500760458e6348fa016194a73}


Free memory occupied by a population of specified parameters. 



Definition at line 105 of file eda.cpp.

Referenced by one\_\-run().

\begin{Code}\begin{verbatim}106 {
107   int i;
108 
109   // free memory used by the population
110 
111   for (i=0; i<N; i++)
112     delete[] x[i];
113 
114   delete[] x;
115 }
\end{verbatim}\end{Code}


\hypertarget{eda_8cpp_68956705a66180170efb2f00c4119486}{
\index{eda.cpp@{eda.cpp}!full_replacement@{full\_\-replacement}}
\index{full_replacement@{full\_\-replacement}!eda.cpp@{eda.cpp}}
\subsubsection[full\_\-replacement]{\setlength{\rightskip}{0pt plus 5cm}int full\_\-replacement (int $\ast$$\ast$ {\em x}, int $\ast$$\ast$ {\em y}, int {\em n}, int {\em N}, double $\ast$ {\em fx}, double $\ast$ {\em fy})}}
\label{eda_8cpp_68956705a66180170efb2f00c4119486}


Replace the entire old population with the new population. 



Definition at line 243 of file eda.cpp.

Referenced by one\_\-run().

\begin{Code}\begin{verbatim}244 {
245   int i;
246 
247   // just replace entire old population with the new guys
248 
249   for (i=0; i<N; i++)
250     {
251       memcpy(x[i],y[i],sizeof(x[i][0])*n);
252       fx[i]=fy[i];
253     }
254 
255   // return the number of processed new individuals
256 
257   return N;
258 }
\end{verbatim}\end{Code}


\hypertarget{eda_8cpp_6691772a0d2ac34de31e1daf06e33343}{
\index{eda.cpp@{eda.cpp}!generate_BB_population@{generate\_\-BB\_\-population}}
\index{generate_BB_population@{generate\_\-BB\_\-population}!eda.cpp@{eda.cpp}}
\subsubsection[generate\_\-BB\_\-population]{\setlength{\rightskip}{0pt plus 5cm}int generate\_\-BB\_\-population (int $\ast$$\ast$ {\em x}, int {\em n}, int {\em N}, int {\em k})}}
\label{eda_8cpp_6691772a0d2ac34de31e1daf06e33343}


Generate a poulation full of blocks of 0s and 1s. Only for debugging. 



Definition at line 57 of file eda.cpp.

References drand().

\begin{Code}\begin{verbatim}58 {
59   int i,j,l;
60 
61   // generate a population full of 000..0 and 111..1 building blocks
62   // (used for testing the model building procedure, unnecessary for
63   // practical applications)
64 
65   for (i=0; i<N; i++)
66     for (j=0; j<n;)
67       {
68         int val=(drand()<0.5)? 0:1;
69         
70         for (l=0; l<k; l++)
71           x[i][j++]=val;
72     };
73 
74   // return the number of generated individuals
75 
76   return N;
77 }
\end{verbatim}\end{Code}


\hypertarget{eda_8cpp_8d2772174edcaff4c414433bb60c440d}{
\index{eda.cpp@{eda.cpp}!generate_population@{generate\_\-population}}
\index{generate_population@{generate\_\-population}!eda.cpp@{eda.cpp}}
\subsubsection[generate\_\-population]{\setlength{\rightskip}{0pt plus 5cm}int generate\_\-population (int $\ast$$\ast$ {\em x}, int {\em n}, int {\em N}, int $\ast$ {\em num\_\-vals} = {\tt NULL})}}
\label{eda_8cpp_8d2772174edcaff4c414433bb60c440d}


Generate a random population of individuals (uniform distribution). 



Definition at line 31 of file eda.cpp.

References drand(), and int\-Rand().

Referenced by one\_\-run().

\begin{Code}\begin{verbatim}32 {
33   int i,j;
34 
35   // general all variables uniformly randomly
36   // (if the numbers of values is not supplied, assumes binary)
37 
38   if (num_vals==NULL)
39     for (i=0; i<n; i++)
40       for (j=0; j<N; j++)
41         x[j][i]=(drand()<0.5)? 0:1;
42   else
43     for (i=0; i<n; i++)
44       for (j=0; j<N; j++)
45         x[j][i]=intRand(num_vals[i]);
46 
47   // return the number of generated individuals
48   
49   return N;
50 }
\end{verbatim}\end{Code}


\hypertarget{eda_8cpp_1329a1829493fea528b0e690ab69f94e}{
\index{eda.cpp@{eda.cpp}!individual_distance@{individual\_\-distance}}
\index{individual_distance@{individual\_\-distance}!eda.cpp@{eda.cpp}}
\subsubsection[individual\_\-distance]{\setlength{\rightskip}{0pt plus 5cm}int individual\_\-distance (int $\ast$ {\em x}, int $\ast$ {\em y}, int {\em n})}}
\label{eda_8cpp_1329a1829493fea528b0e690ab69f94e}


Compute a distance between two individuals (number of non-matching chars). 



Definition at line 265 of file eda.cpp.

Referenced by restricted\_\-tournament\_\-replacement().

\begin{Code}\begin{verbatim}266 {
267   int i;
268   int d=0;
269 
270   for (i=0; i<n; i++)
271     if (x[i]!=y[i])
272       d++;
273 
274   return d;
275 }
\end{verbatim}\end{Code}


\hypertarget{eda_8cpp_334f5d73c2118df22dd59546965be64c}{
\index{eda.cpp@{eda.cpp}!one_run@{one\_\-run}}
\index{one_run@{one\_\-run}!eda.cpp@{eda.cpp}}
\subsubsection[one\_\-run]{\setlength{\rightskip}{0pt plus 5cm}int one\_\-run (\hyperlink{struct_parameters}{Parameters} $\ast$ {\em params}, \hyperlink{struct_population_statistics}{Population\-Statistics} $\ast$ {\em stats})}}
\label{eda_8cpp_334f5d73c2118df22dd59546965be64c}


Execute one run of a decision-tree EDA with the specified parameters. 



Definition at line 282 of file eda.cpp.

References allocate\_\-population(), Parameters::bisection, compute\_\-population\_\-statistics(), evaluate\_\-population(), free\_\-population(), full\_\-replacement(), generate\_\-population(), init\_\-population\_\-statistics(), Parameters::max\_\-generations, N, Population\-Statistics::num\_\-evals, Parameters::population\_\-size, print\_\-status(), Parameters::problem\_\-size, Parameters::quiet\_\-mode, Parameters::replacement, restricted\_\-tournament\_\-replacement(), set\_\-num\_\-vals(), Population\-Statistics::success, tournament\_\-selection(), Parameters::tournament\_\-size, variation(), and Parameters::verbose\_\-mode.

Referenced by do\_\-runs(), and main().

\begin{Code}\begin{verbatim}283 {
284   int N=params->population_size;
285   int n=params->problem_size;
286   int max_generations=params->max_generations;
287 
288   int **current_population;
289   int **selected_population;
290   int **sampled_population;
291 
292   double *current_f;
293   double *sampled_f;
294 
295   int *num_vals = new int[n];
296 
297   // allocate populations and necessary fitness arrays
298 
299   current_population = allocate_population(n,N);
300   selected_population = allocate_population(n,N);
301   sampled_population = allocate_population(n,N);
302   
303   current_f = new double[N];
304   sampled_f = new double[N];
305 
306   // initialize the replacement method
307 
308   ReplacementMethod *replacement;
309   if (params->replacement==0)
310     replacement=restricted_tournament_replacement;
311   else
312     if (params->replacement==1)
313       replacement=full_replacement;
314     else
315       {
316         printf("ERROR: Unknown replacement method (%u)\n",params->replacement);
317         exit(-2);
318       };
319 
320   // initialize the number of values
321 
322   set_num_vals(num_vals,n);
323 
324   // generate initial population
325 
326   generate_population(current_population,n,N,num_vals);
327 
328   // evaluate initial population
329 
330   evaluate_population(current_population,current_f,n,N,stats->num_evals);
331 
332   // initialize some main-loop variables
333 
334   int done=0;
335   int t=0;
336 
337   // init the statistics and print the initial status information
338 
339   init_population_statistics(stats);
340   compute_population_statistics(current_population,current_f,n,N,num_vals,stats);
341   
342   if (params->quiet_mode==0)
343     print_status(t,stats);
344   else
345     params->verbose_mode=0;
346 
347   // main loop
348 
349   while (!done)
350     {
351       // increment generation counter
352 
353       t++;
354 
355       // selection
356 
357       tournament_selection(selected_population,
358                            current_population,
359                            current_f,
360                            n,
361                            N,
362                            params->tournament_size);
363 
364       // variation
365 
366       variation(sampled_population,selected_population,params,num_vals);
367 
368       // evaluation of new candidates
369       
370       evaluate_population(sampled_population,sampled_f,n,N,stats->num_evals);
371       
372       // replacement
373       
374       replacement(current_population,
375                   sampled_population,
376                   n,
377                   N,
378                   current_f,
379                   sampled_f);
380 
381       // statistics
382 
383       compute_population_statistics(current_population,current_f,n,N,num_vals,stats);
384 
385       // print current status
386 
387       if ((params->quiet_mode==0)&&(params->bisection==0))
388         print_status(t,stats);
389 
390       // should terminate?
391 
392       if ((t>max_generations)||(stats->success==1))
393         done=1;
394     }
395 
396   // free memory
397 
398   delete[] num_vals;
399   delete[] sampled_f;
400   delete[] current_f;
401 
402   free_population(current_population,n,N);
403   free_population(selected_population,n,N);
404   free_population(sampled_population,n,N);
405 
406   // return with success/failure (0/1)
407 
408   return stats->success;
409 }
\end{verbatim}\end{Code}


\hypertarget{eda_8cpp_aea6e51089cae913316f55a552b9226c}{
\index{eda.cpp@{eda.cpp}!print_population@{print\_\-population}}
\index{print_population@{print\_\-population}!eda.cpp@{eda.cpp}}
\subsubsection[print\_\-population]{\setlength{\rightskip}{0pt plus 5cm}void print\_\-population (FILE $\ast$ {\em f}, int $\ast$$\ast$ {\em x}, int {\em n}, int {\em N})}}
\label{eda_8cpp_aea6e51089cae913316f55a552b9226c}


Print the population to the specified file. 



Definition at line 175 of file eda.cpp.

References objective\_\-function().

\begin{Code}\begin{verbatim}176 {
177   int i,j;
178   
179   for (i=0; i<N; i++)
180     {
181       for (j=0; j<n; j++)
182         fprintf(f,"%u",x[i][j]);
183       fprintf(f,"   %f\n",objective_function(x[i],n));
184     };
185 }
\end{verbatim}\end{Code}


\hypertarget{eda_8cpp_be5265230638c7fb37140688226999e2}{
\index{eda.cpp@{eda.cpp}!print_status@{print\_\-status}}
\index{print_status@{print\_\-status}!eda.cpp@{eda.cpp}}
\subsubsection[print\_\-status]{\setlength{\rightskip}{0pt plus 5cm}void print\_\-status (int {\em t}, \hyperlink{struct_population_statistics}{Population\-Statistics} $\ast$ {\em stats})}}
\label{eda_8cpp_be5265230638c7fb37140688226999e2}


Print the status of the algorithm. 



Definition at line 451 of file eda.cpp.

References Population\-Statistics::avg\-F, Population\-Statistics::max\-F, Population\-Statistics::min\-F, and separator().

Referenced by one\_\-run().

\begin{Code}\begin{verbatim}452 {
453   printf("Generation: %u\n",t);
454   printf("   min:  %f\n",stats->minF);
455   printf("   max:  %f\n",stats->maxF);
456   printf("   mean: %f\n",stats->avgF);
457   separator(stdout);
458 }
\end{verbatim}\end{Code}


\hypertarget{eda_8cpp_83b5f930be040dac41889ae6d326a47a}{
\index{eda.cpp@{eda.cpp}!print_summary@{print\_\-summary}}
\index{print_summary@{print\_\-summary}!eda.cpp@{eda.cpp}}
\subsubsection[print\_\-summary]{\setlength{\rightskip}{0pt plus 5cm}void print\_\-summary (\hyperlink{struct_population_statistics}{Population\-Statistics} $\ast$ {\em stats})}}
\label{eda_8cpp_83b5f930be040dac41889ae6d326a47a}


Print summary of a run. 



Definition at line 478 of file eda.cpp.

References Population\-Statistics::max\-F, Population\-Statistics::num\_\-evals, Population\-Statistics::population\_\-size, Population\-Statistics::problem\_\-size, and Population\-Statistics::success.

Referenced by main().

\begin{Code}\begin{verbatim}479 {
480   printf("Summary:\n");
481   printf("   status       = %s\n",(stats->success)? "success":"failure");
482   printf("   best_found   = %f\n",stats->maxF);
483   printf("   num_evals    = %lu\n",stats->num_evals);
484   printf("   pop_size     = %u\n",stats->population_size);
485   printf("   problem_size = %u\n",stats->problem_size);
486 }
\end{verbatim}\end{Code}


\hypertarget{eda_8cpp_f277260af3b0a29adf95a57fe3581404}{
\index{eda.cpp@{eda.cpp}!restricted_tournament_replacement@{restricted\_\-tournament\_\-replacement}}
\index{restricted_tournament_replacement@{restricted\_\-tournament\_\-replacement}!eda.cpp@{eda.cpp}}
\subsubsection[restricted\_\-tournament\_\-replacement]{\setlength{\rightskip}{0pt plus 5cm}int restricted\_\-tournament\_\-replacement (int $\ast$$\ast$ {\em x}, int $\ast$$\ast$ {\em y}, int {\em n}, int {\em N}, double $\ast$ {\em fx}, double $\ast$ {\em fy})}}
\label{eda_8cpp_f277260af3b0a29adf95a57fe3581404}


Restricted tournament replacement (Harik, 1995) for niching. 



Definition at line 192 of file eda.cpp.

References individual\_\-distance(), and int\-Rand().

Referenced by one\_\-run().

\begin{Code}\begin{verbatim}193 {
194   int i,j;
195 
196   // use default value for window size
197 
198   int windowSize=(n<N/20)? n:(N/20);
199 
200   // for every individual, do the same
201 
202   for (i=0; i<N; i++)
203     {
204       // select a random subset from the original population (window) and 
205       // find the most similar guy to the new individual in this window
206       // (string Hamming distance)
207 
208       int pick=intRand(N);
209       int dist=individual_distance(x[i],y[pick],n);
210 
211       for (j=1; j<windowSize; j++)
212         {
213           int pick2=intRand(N);
214           int dist2=individual_distance(x[i],y[pick2],n);
215 
216           if (dist2<dist)
217             {
218               pick=pick2;
219               dist=dist2;
220             }
221         }
222 
223       // if the most similar guy from the window is better than the new guy,
224       // the new guy replaces it
225 
226       if (fx[i]<fy[pick])
227         {
228           memcpy(x[i],y[pick],sizeof(x[i][0])*n);
229           fx[i]=fy[pick];
230         }
231     }
232 
233   // return the number of processed new individuals
234 
235   return N;
236 }
\end{verbatim}\end{Code}


\hypertarget{eda_8cpp_ac7bc3c5ce16c23728f6ed1bd6534f56}{
\index{eda.cpp@{eda.cpp}!separator@{separator}}
\index{separator@{separator}!eda.cpp@{eda.cpp}}
\subsubsection[separator]{\setlength{\rightskip}{0pt plus 5cm}void separator (FILE $\ast$ {\em f}, int {\em type})}}
\label{eda_8cpp_ac7bc3c5ce16c23728f6ed1bd6534f56}


Print a sequence of dashes to separate text output. 



Definition at line 465 of file eda.cpp.

Referenced by bisection(), print\_\-parameters(), print\_\-status(), and variation().

\begin{Code}\begin{verbatim}466 {
467   if (type==0)
468     fprintf(f,"- - - - - - - - - - - - - - - - - - - - - - - - - - - - - - - - - - - - - -\n");
469   else
470     fprintf(f,"---------------------------------------------------------------------------\n");
471 }
\end{verbatim}\end{Code}


\hypertarget{eda_8cpp_b7834469dba3fdd4d1e73b368051dbae}{
\index{eda.cpp@{eda.cpp}!tournament_selection@{tournament\_\-selection}}
\index{tournament_selection@{tournament\_\-selection}!eda.cpp@{eda.cpp}}
\subsubsection[tournament\_\-selection]{\setlength{\rightskip}{0pt plus 5cm}int tournament\_\-selection (int $\ast$$\ast$ {\em y}, int $\ast$$\ast$ {\em x}, double $\ast$ {\em f}, int {\em n}, int {\em N}, int {\em k})}}
\label{eda_8cpp_b7834469dba3fdd4d1e73b368051dbae}


Tournament selection with replacement (k-ary tournaments). 



Definition at line 143 of file eda.cpp.

References int\-Rand().

Referenced by one\_\-run().

\begin{Code}\begin{verbatim}144 {
145   int i,j;
146 
147   for (i=0; i<N; i++)
148     {
149       // select a winner of k tournaments (with replacement)
150 
151       int winner=intRand(N);
152       for (j=1; j<k; j++)
153         {
154           int l=intRand(N);
155           if (f[l]>f[winner])
156             winner=l;
157         };
158 
159       // the winner takes the next spot in the selected population
160 
161       for (int ii=0; ii<n; ii++)
162         y[i][ii]=x[winner][ii];
163     };
164 
165   // return the number of selected individuals
166 
167   return N;
168 }
\end{verbatim}\end{Code}


\hypertarget{eda_8cpp_8eef87c9ae12bcae144b6b020865df42}{
\index{eda.cpp@{eda.cpp}!variation@{variation}}
\index{variation@{variation}!eda.cpp@{eda.cpp}}
\subsubsection[variation]{\setlength{\rightskip}{0pt plus 5cm}int variation (int $\ast$$\ast$ {\em sampled\_\-population}, int $\ast$$\ast$ {\em selected\_\-population}, \hyperlink{struct_parameters}{Parameters} $\ast$ {\em params}, int $\ast$ {\em num\_\-vals})}}
\label{eda_8cpp_8eef87c9ae12bcae144b6b020865df42}


Variation operator of the decision-tree EDA (learns and samples model). 



Definition at line 416 of file eda.cpp.

References Tree\-Model::learn\-Probabilities(), Tree\-Model::learn\-Structure(), N, Parameters::population\_\-size, Parameters::problem\_\-size, Tree\-Model::sample\-Model(), separator(), and Parameters::verbose\_\-mode.

Referenced by one\_\-run().

\begin{Code}\begin{verbatim}417 {
418   int N=params->population_size;
419   int n=params->problem_size;
420   int loud=params->verbose_mode;
421 
422   if (loud)
423     separator(stdout,0);
424 
425   TreeModel *t = new TreeModel(n, num_vals);
426  
427   // learn the model structure and model parameters
428 
429   t->learnStructure(selected_population,N,loud);
430   t->learnProbabilities(selected_population,N,loud);
431 
432   // sample the learned model to generate new candidate solutions
433 
434   t->sampleModel(sampled_population,N);
435 
436   // free memory
437 
438   delete t;
439 
440   // return the number of generated individuals
441 
442   return N;
443 }
\end{verbatim}\end{Code}


