\hypertarget{parse-input_8hpp}{
\section{parse-input.hpp File Reference}
\label{parse-input_8hpp}\index{parse-input.hpp@{parse-input.hpp}}
}
Header file for \hyperlink{parse-input_8cpp}{parse-input.cpp}. 

{\tt \#include $<$stdio.h$>$}\par
{\tt \#include \char`\"{}eda.hpp\char`\"{}}\par
\subsection*{Functions}
\begin{CompactItemize}
\item 
void \hyperlink{parse-input_8hpp_ecc496a88afbe998e2577dfe766dbd0e}{parse\_\-input\_\-file} (FILE $\ast$f, \hyperlink{struct_parameters}{Parameters} $\ast$params)
\begin{CompactList}\small\item\em Parse the input file and store the results in a structure \hyperlink{struct_parameters}{Parameters}. \item\end{CompactList}\item 
void \hyperlink{parse-input_8hpp_e30f981622da23c664e95e813953673d}{param\_\-help} ()
\begin{CompactList}\small\item\em Print a short description of parameters. \item\end{CompactList}\item 
int \hyperlink{parse-input_8hpp_e28f089803e889dc1c1412f56f482ee7}{get\_\-new\_\-identifier} (FILE $\ast$f, char $\ast$s)
\begin{CompactList}\small\item\em Get a new identifier from the file and scan until after next '='. \item\end{CompactList}\item 
int \hyperlink{parse-input_8hpp_84d83813df0dc09fd6a2a4fbfc893d03}{get\_\-int\_\-value} (FILE $\ast$f, int \&val)
\begin{CompactList}\small\item\em Read an integer value from a file. \item\end{CompactList}\item 
int \hyperlink{parse-input_8hpp_9d6eb3d840afb4c96f4d0e52e61c3bf7}{print\_\-parameters} (\hyperlink{struct_parameters}{Parameters} $\ast$params)
\begin{CompactList}\small\item\em Print the parameters to stdout. \item\end{CompactList}\end{CompactItemize}


\subsection{Detailed Description}
Header file for \hyperlink{parse-input_8cpp}{parse-input.cpp}. 



Definition in file \hyperlink{parse-input_8hpp-source}{parse-input.hpp}.

\subsection{Function Documentation}
\hypertarget{parse-input_8hpp_84d83813df0dc09fd6a2a4fbfc893d03}{
\index{parse-input.hpp@{parse-input.hpp}!get_int_value@{get\_\-int\_\-value}}
\index{get_int_value@{get\_\-int\_\-value}!parse-input.hpp@{parse-input.hpp}}
\subsubsection[get\_\-int\_\-value]{\setlength{\rightskip}{0pt plus 5cm}int get\_\-int\_\-value (FILE $\ast$ {\em f}, int \& {\em val})}}
\label{parse-input_8hpp_84d83813df0dc09fd6a2a4fbfc893d03}


Read an integer value from a file. 



Definition at line 166 of file parse-input.cpp.

References MAX\_\-LINE\_\-WIDTH.

Referenced by parse\_\-input\_\-file().

\begin{Code}\begin{verbatim}167 {
168   char line[MAX_LINE_WIDTH];
169   fgets(line,MAX_LINE_WIDTH,f);
170   sscanf(line,"%u",&val);
171 
172   return 1;
173 }
\end{verbatim}\end{Code}


\hypertarget{parse-input_8hpp_e28f089803e889dc1c1412f56f482ee7}{
\index{parse-input.hpp@{parse-input.hpp}!get_new_identifier@{get\_\-new\_\-identifier}}
\index{get_new_identifier@{get\_\-new\_\-identifier}!parse-input.hpp@{parse-input.hpp}}
\subsubsection[get\_\-new\_\-identifier]{\setlength{\rightskip}{0pt plus 5cm}int get\_\-new\_\-identifier (FILE $\ast$ {\em f}, char $\ast$ {\em s})}}
\label{parse-input_8hpp_e28f089803e889dc1c1412f56f482ee7}


Get a new identifier from the file and scan until after next '='. 



Definition at line 105 of file parse-input.cpp.

Referenced by parse\_\-input\_\-file().

\begin{Code}\begin{verbatim}106 {
107   char c=0;
108   int n=0;
109   int done=0;
110 
111   while ((!feof(f))&&(!done))
112     {
113       c=fgetc(f);
114       if (((c>='a')&&(c<='z'))||
115           ((c>='A')&&(c<='Z'))||
116           (c=='_'))
117         done=1;
118     }
119 
120   if (done)
121     {
122       s[0]=c;
123       n=1;
124       done=0;
125       while ((!feof(f))&&(!done)) 
126         {
127           c=fgetc(f);
128           if (((c>='a')&&(c<='z'))||
129               ((c>='A')&&(c<='Z'))||
130               (c=='_'))
131             s[n++]=c;
132           else 
133             done=1;
134         };
135       s[n]='\0';
136 
137       if (c=='=')
138         done=1;
139       else
140         done=0;
141 
142       while ((!feof(f))&&(!done))
143         {
144           c=fgetc(f);
145           if (c=='=')
146             done=1;
147         };
148       
149       if (!done)
150         {
151           printf("Expected '=' after identifier %s\n",s);
152           exit(-1);
153         }
154 
155       return 1;
156     }
157 
158   return 0;
159 }
\end{verbatim}\end{Code}


\hypertarget{parse-input_8hpp_e30f981622da23c664e95e813953673d}{
\index{parse-input.hpp@{parse-input.hpp}!param_help@{param\_\-help}}
\index{param_help@{param\_\-help}!parse-input.hpp@{parse-input.hpp}}
\subsubsection[param\_\-help]{\setlength{\rightskip}{0pt plus 5cm}void param\_\-help ()}}
\label{parse-input_8hpp_e30f981622da23c664e95e813953673d}


Print a short description of parameters. 



Definition at line 83 of file parse-input.cpp.

Referenced by main().

\begin{Code}\begin{verbatim}84 {
85   printf("List of parameters for input file (you can specify any subset):\n\n");
86 
87   printf("  -> Population size:\n      population_size = <number>\n\n");
88   printf("  -> Problem size (number of characters):\n      problem_size = <number>\n\n");
89   printf("  -> Maximum number of generations:\n      max_generations = <number>\n\n");
90   printf("  -> Replacement (0=restricted tournament repl., 1=full repl.):\n      replacement = <number>\n\n");
91   printf("  -> Tournament size for tournament selection:\n      tournament_size = <number>\n\n");
92   printf("  -> Use bisection to optimize the population size:\n      bisection = <number>\n\n");
93   printf("  -> Number of successful runs for optimal population sizing:\n      num_runs=<number>\n\n");
94   printf("  -> Quiet mode (prints only the end-of-the-run summary):\n      quiet_mode = <number>\n\n");
95   printf("  -> Verbose mode (do not use with redirected output):\n      verbose_mode = <number>\n\n");
96   printf("  -> Random seed:\n      random_seed = <number>\n\n");
97   printf("See example_input, example_input_bisection, and example_input_big for example input files\n");
98 }
\end{verbatim}\end{Code}


\hypertarget{parse-input_8hpp_ecc496a88afbe998e2577dfe766dbd0e}{
\index{parse-input.hpp@{parse-input.hpp}!parse_input_file@{parse\_\-input\_\-file}}
\index{parse_input_file@{parse\_\-input\_\-file}!parse-input.hpp@{parse-input.hpp}}
\subsubsection[parse\_\-input\_\-file]{\setlength{\rightskip}{0pt plus 5cm}void parse\_\-input\_\-file (FILE $\ast$ {\em f}, \hyperlink{struct_parameters}{Parameters} $\ast$ {\em params})}}
\label{parse-input_8hpp_ecc496a88afbe998e2577dfe766dbd0e}


Parse the input file and store the results in a structure \hyperlink{struct_parameters}{Parameters}. 

\begin{Desc}
\item[See also:]Example input parameter file: \hyperlink{example__input-example}{example\_\-input} \end{Desc}


Definition at line 23 of file parse-input.cpp.

References Parameters::bisection, get\_\-int\_\-value(), get\_\-new\_\-identifier(), Parameters::max\_\-generations, MAX\_\-LINE\_\-WIDTH, Parameters::num\_\-bisection\_\-runs, Parameters::population\_\-size, Parameters::problem\_\-size, Parameters::quiet\_\-mode, Parameters::replacement, set\-Seed(), Parameters::tournament\_\-size, and Parameters::verbose\_\-mode.

Referenced by main().

\begin{Code}\begin{verbatim}24 {
25   if (f==NULL)
26     return;
27 
28   char id[MAX_LINE_WIDTH];
29 
30   int seed_set=0;
31   while (get_new_identifier(f,id))
32     {
33       if (strcmp(id,"population_size")==0)
34         get_int_value(f,params->population_size);
35       else
36       if (strcmp(id,"max_generations")==0)
37         get_int_value(f,params->max_generations);
38       else
39       if (strcmp(id,"problem_size")==0)
40         get_int_value(f,params->problem_size);
41       else
42       if (strcmp(id,"replacement")==0)
43         get_int_value(f,params->replacement);
44       else
45       if (strcmp(id,"tournament_size")==0)
46         get_int_value(f,params->tournament_size);
47       else
48       if (strcmp(id,"bisection")==0)
49         get_int_value(f,params->bisection);
50       else
51       if (strcmp(id,"num_bisection_runs")==0)
52         get_int_value(f,params->num_bisection_runs);
53       else
54       if (strcmp(id,"quiet_mode")==0)
55         get_int_value(f,params->quiet_mode);
56       else
57       if (strcmp(id,"verbose_mode")==0)
58         get_int_value(f,params->verbose_mode);
59       else
60       if (strcmp(id,"random_seed")==0)
61         {
62           int seed;
63           get_int_value(f,seed);
64           setSeed(seed);
65           seed_set=1;
66         }
67       else
68         {
69           printf("%s is an unknown identifier\n",id);
70           exit(-1);
71         };
72     };
73   
74   if (seed_set==0)
75     setSeed(123);
76 }
\end{verbatim}\end{Code}


\hypertarget{parse-input_8hpp_9d6eb3d840afb4c96f4d0e52e61c3bf7}{
\index{parse-input.hpp@{parse-input.hpp}!print_parameters@{print\_\-parameters}}
\index{print_parameters@{print\_\-parameters}!parse-input.hpp@{parse-input.hpp}}
\subsubsection[print\_\-parameters]{\setlength{\rightskip}{0pt plus 5cm}int print\_\-parameters (\hyperlink{struct_parameters}{Parameters} $\ast$ {\em params})}}
\label{parse-input_8hpp_9d6eb3d840afb4c96f4d0e52e61c3bf7}


Print the parameters to stdout. 



Definition at line 180 of file parse-input.cpp.

References Parameters::bisection, Parameters::max\_\-generations, Parameters::num\_\-bisection\_\-runs, Parameters::population\_\-size, Parameters::problem\_\-size, Parameters::quiet\_\-mode, Parameters::replacement, separator(), Parameters::tournament\_\-size, and Parameters::verbose\_\-mode.

Referenced by main().

\begin{Code}\begin{verbatim}181 {
182   printf("Parameters:\n");
183   printf("  population_size    = %u\n",params->population_size);
184   printf("  problem_size       = %u\n",params->problem_size);
185   printf("  max_generations    = %u\n",params->max_generations);
186   printf("  tournament_size    = %u\n",params->tournament_size);
187   printf("  replacement        = %u\n",params->replacement);
188   printf("  bisection          = %u\n",params->bisection);
189   printf("  num_bisection_runs = %u\n",params->num_bisection_runs);
190   printf("  quiet_mode         = %u\n",params->quiet_mode);
191   printf("  verbose_mode       = %u\n",params->verbose_mode);
192   separator(stdout);
193 
194   return 0;
195 }
\end{verbatim}\end{Code}


